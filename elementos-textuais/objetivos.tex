\chapter{OBJETIVOS}

A educação em Ciência da Computação e Robótica tem um papel estratégico na formação de estudantes capazes de lidar com os desafios tecnológicos do século XXI. Nesse contexto, o projeto "Prognitive"  se apresenta como uma resposta proativa às necessidades educacionais contemporâneas, com a proposta de ir além do ensino tradicional, promovendo um aprendizado prático, personalizado e interdisciplinar. Neste capítulo, serão apresentados os objetivos traçados pelo projeto e as principais ações desenvolvidas para sua concretização.

\section{Objetivos Propostos e Discussões das Ações Desenvolvidas}

O projeto "Prognitive" estabelece como objetivos centrais:

\begin{itemize}
    \item Ministrar aulas de programação com a linguagem Java, fundamentadas nos princípios da Orientação a Objetos, integrando esses conhecimentos aos conceitos da robótica educacional, com o intuito de desenvolver competências técnicas e estruturadas em programação;
    \item Estimular a aprendizagem ativa por meio de exercícios de fixação, favorecendo o desenvolvimento do raciocínio lógico e da construção de algoritmos, aspectos essenciais na formação em computação;
    \item Oferecer acompanhamento individualizado aos alunos, promovendo o esclarecimento de dúvidas e o apoio na execução das atividades, assegurando um processo de aprendizagem mais eficaz e ajustado às particularidades de cada estudante;
    \item Inserir e aplicar a ferramenta Road Runner, convertendo a arena em um plano cartesiano, o que possibilita um controle mais preciso do robô na fase autônoma, inclusive em trajetórias não lineares, ampliando as habilidades de programação e estratégias de movimentação;
    \item Preparar os discentes para competir na First Tech Challenge (FTC), competição nacional e internacional de robótica educacional, oferecendo base técnica e científica, experiência prática e incentivo ao trabalho em equipe e à resolução de problemas;
    \item Estimular o interesse dos alunos pelas áreas de Ciência, Tecnologia, Engenharia e Matemática (STEM), contribuindo para a formação de futuros profissionais nessas áreas estratégicas para o desenvolvimento tecnológico.
\end{itemize}

Diversas ações foram implementadas em diferentes cenários, utilizando abordagens dinâmicas e focadas no engajamento estudantil:

\begin{itemize}
    \item Realização de aulas práticas de programação com a linguagem Java, aliadas à robótica com foco em conceitos de Orientação a Objetos;
    \item Proposição de exercícios de fixação com desafios práticos, reforçando o conteúdo e promovendo o pensamento computacional;
    \item Atendimento contínuo aos alunos, com suporte direto na resolução de atividades e dúvidas durante as práticas;
    \item Introdução e aplicação da ferramenta Road Runner, permitindo que os alunos programem trajetórias com base em coordenadas cartesianas e compreendam os efeitos de suas escolhas na movimentação real do robô;
    \item Preparação para participação na FTC, com atividades direcionadas para treino da competição e desenvolver estratégias em equipe.
    \item Prestação de mentoria na construção científica do portfólio da equipe, contribuindo com orientações quanto à organização, redação técnica e apresentação das atividades desenvolvidas, fortalecendo a identidade e a comunicação do projeto.
\end{itemize}

\section{Objetivos Alcançados}

Os objetivos delineados pelo projeto "Prognitive" foram atingidos de forma expressiva e produtiva. Todas as ações planejadas foram executadas com êxito, proporcionando aos alunos um ambiente propício ao desenvolvimento técnico em programação e robótica. O suporte individualizado permitiu que cada estudante avançasse conforme seu próprio ritmo, enfrentando e superando dificuldades com segurança e motivação. A utilização da ferramenta \textit{Road Runner} e o treinamento voltado para a FTC contribuíram significativamente para a consolidação das habilidades práticas e do pensamento estratégico. Como fruto desse esforço, a equipe de Robótica Legonautas participou da Temporada 2023/2024 da \textit{FIRST Tech Challenge}, conquistando o título de Vice-Campeões na Arena e sendo premiados como Aliança Capitã Finalista. Na temporada 2024/2025, a equipe também chegou às etapas finais na arena do campeonato. Além disso, observou-se um notável aumento no interesse dos alunos por áreas STEM, demonstrando que o projeto cumpriu com êxito seu propósito de inspirar e formar jovens para os desafios tecnológicos do presente e do futuro.
