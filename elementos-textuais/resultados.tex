\chapter{RESULTADOS E CONTRIBUIÇÕES DO PROJETO À COMUNIDADE}

O capítulo de Resultados e Contribuições do Projeto à Comunidade apresenta uma análise abrangente dos impactos e benefícios gerados pela execução do "Prognitive" em sua comunidade-alvo. Ao longo do período de implementação, uma série de resultados positivos foi observada, evidenciando a eficácia e relevância das iniciativas propostas.

Um dos resultados mais notáveis foi o aumento significativo do interesse dos alunos pelas áreas de Computação e STEM. Essa crescente motivação foi percebida tanto na participação ativa dos estudantes nas atividades do projeto quanto em seu engajamento em buscar conhecimento adicional fora do ambiente escolar. Esse incremento no interesse demonstra o impacto direto do projeto na promoção da educação e formação de futuros profissionais em áreas tecnológicas.

Além disso, os participantes do projeto apresentaram uma melhoria substancial em suas habilidades de programação e raciocínio lógico. Isso se refletiu não apenas em avaliações internas e externas, onde os alunos obtiveram resultados expressivos, mas também em sua capacidade aprimorada de resolver problemas de forma eficiente e criativa. Essa habilidade é crucial não apenas para o sucesso acadêmico, mas também para a preparação dos jovens para os desafios do mercado de trabalho.

Outro aspecto relevante dos resultados obtidos foi a participação destacada dos alunos em competições de robótica e feiras de ciências  (Fig. \ref{fig:1} e Fig.\ref{fig:3}). Nessas ocasiões, os estudantes puderam aplicar os seus conhecimentos adquiridos no projeto de forma prática, demonstrando suas habilidades para a comunidade e contribuindo para a visibilidade da escola e do projeto no cenário local, regional e nacional. Nesse sentido, destaca-se a participação da equipe de robótica Legonautas na Competição FTC, Temporada 2023/2024 e 2024/2025 da FTC e foram Vice Campeões na Arena, ganhando o prêmio de Aliança Capitã Finalista. 

\begin{figure}[H]
    \centering
    \includegraphics[scale=0.8]{imagens/imagem9.jpeg}
    \caption{Participação dos Alunos atendidos pelo projeto na Competição Nacional da FTC em Brasília.}
    \label{fig:1}
\end{figure}

Além dos resultados individuais dos participantes, o "Prognitive" também trouxe contribuições significativas para a comunidade em geral. Em primeiro lugar, o projeto estimulou o desenvolvimento tecnológico local, ao formar uma nova geração de estudantes capacitados em áreas como programação e robótica. Esses jovens talentosos têm o potencial de contribuir significativamente para a inovação e o crescimento econômico da região.

O projeto fortaleceu o vínculo entre a escola e a comunidade, ao envolver pais, responsáveis e outros membros da sociedade nas atividades do projeto. Essa colaboração entre diferentes atores sociais é essencial para promover uma educação de qualidade e para preparar os jovens para os desafios do século XXI.

O "Prognitive" proporcionou oportunidades para os alunos desenvolverem habilidades de liderança e trabalho em equipe. A participação em projetos de programação e robótica exigiu que os estudantes trabalhassem juntos para alcançar objetivos comuns, preparando-os para desafios futuros no ambiente de trabalho (Fig. \ref{fig:3}). 

Além de aprimorar as habilidades técnicas em programação e robótica, o projeto também proporcionou aos alunos oportunidades para o desenvolvimento de habilidades de liderança, trabalho em equipe e resolução de problemas em grupo. A participação em competições de robótica e no desenvolvimento de projetos exigiu uma intensa colaboração entre os estudantes para o desenvolvimento de soluções criativas em conjunto.

\begin{figure}[H]
    \centering
    \includegraphics[scale=0.6]{imagens/imagem8.jpeg}
    \caption{Preparação do Robô para Competição utilizando Road Runner.}
    \label{fig:3}
\end{figure}

\begin{figure}[H]
    \centering
    \includegraphics[scale=0.22]{imagens/imagem 20.jpg}
    \caption{Prêmio de Aliança Finalista - FTC.}
    \label{fig:4}
\end{figure}

Por fim, o projeto teve um impacto significativo e duradouro no desenvolvimento do pensamento crítico e criativo dos alunos e da comunidade. Ao serem desafiados a resolver problemas complexos e desenvolver soluções inovadoras através da programação e da robótica, os participantes aprimoraram suas habilidades de análise, tomada de decisões e resolução de problemas. Essas competências são fundamentais não só para a carreira profissional, mas também para a formação de cidadãos engajados e capazes de contribuir positivamente para a sociedade.
