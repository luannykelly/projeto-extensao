\chapter{IDENTIFICAÇÃO DA COMUNIDADE PARTICIPANTE NA AÇÃO DE EXTENSÃO}

As ações extensionistas do projeto foram direcionadas principalmente na escola Dionízio Marques de Almeida, visando atingir um público diversificado e promover o acesso à educação em ciência da computação e robótica.

Durante a execução do projeto, foram estabelecidos os seguintes convênios, contratos e acordos de cooperação:

\begin{itemize}
\item \textbf{Convênios celebrados:}
\begin{itemize}
\item Instituições públicas federais: 0
\item Instituições públicas estaduais: 0
\item Instituições públicas municipais: 0
\item Movimentos Sociais Organizados: 0
\end{itemize}
\item \textbf{Contratos firmados:}
\begin{itemize}
    \item Instituições públicas federais: 0
    \item Instituições públicas estaduais: 0
    \item Instituições públicas municipais: 0
    \item Movimentos Sociais Organizados: 0
\end{itemize}

\item \textbf{Acordos de cooperação estabelecidos:}
\begin{itemize}
    \item Instituições públicas federais: 0
    \item Instituições públicas estaduais: 0
    \item Instituições públicas municipais: 0
    \item Movimentos Sociais Organizados: 0
\end{itemize}

\end{itemize}

Além disso, registrou-se o número de pessoas externas à UPEB (Universidade Pública Estadual da Baixada) participantes/atendidas pelo projeto, curso ou evento:

\begin{enumerate}
\item \textbf{Da Comunidade:}
\begin{itemize}
\item Quantidade: 40
\end{itemize}

\item \textbf{Como Colaborador(a):}
\begin{itemize}
    \item Quantidade: 2
\end{itemize}

\item \textbf{Carga horária:}
\begin{itemize}
    \item Semanal: 4
    \item Mensal: 20
    \item Total: 360
\end{itemize}
\end{enumerate}

Este registro inclui tanto a participação/atendimento durante as atividades desenvolvidas, planejamento das ações, como também aquelas realizadas por meio da prestação de serviços.
