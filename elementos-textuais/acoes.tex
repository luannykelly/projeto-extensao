\chapter{IDENTIFICAÇÃO DAS AÇÕES DESENVOLVIDAS}

Para garantir o cumprimento dos objetivos do projeto Prognitive, foram realizadas diversas ações ao longo de sua execução. As atividades foram conduzidas semanalmente, no Laboratório de Informática da Escola Dionísio Marques de Almeida. A seguir, apresentam-se as datas e os conteúdos abordados em cada encontro, organizados de forma progressiva para preparar os estudantes tanto no domínio da programação quanto na atuação em robótica competitiva, com foco em Java e no uso de tecnologias como Road Runner.

\vspace{1em}

\textbf{Momento I – Introdução ã Lógica de Programação}

\begin{itemize}
    \item 15/08/24: Apresentação do projeto e introdução ao pensamento computacional
    \item 19/08/24: Introdução ã Lógica de Programação
    \item 22/08/24: Introdução à Linguagem Java
    \item 29/08/24:  Tipos de dados, constantes e variáveis
\end{itemize}

\textbf{Momento II – Fundamentos de Programação em Java}

\begin{itemize}
    \item 05/09/24: Entrada e saída de dados com a classe Scanner
    \item 12/09/24: Correção coletiva de atividades práticas
    \item 19/09/24: Comandos de seleção e comandos de repetição (if e else, while e for)
    \item 26/09/24: Vetores, matrizes e manipulação de strings
    \item 03/10/24: Funções, Procedimentos e Recursividade
    \item 10/10/24: Registros e Enumerações
\end{itemize}

\textbf{Momento III – Programação Orientada a Objetos}

\begin{itemize}
    \item 17/10/24: Introdução à Programação Orientada a Objetos: 
    \item 24/10/24: Classes, Objetos, Atributos e Métodos
    \item 31/10/24: Objetos e Construtores
    \item 07/11/24: Composição e Agregação 
    \item 14/11/24: Encapsulamento e Abstração
    \item 21/11/24: Herança e Polimorfismo
    \item 28/11/24: Coleções em Java
    \item 05/12/24: Tratamento e Lançamento de Exceções 
    \item 12/12/24: Classes Abstratas e Interfaces
\end{itemize}

\textbf{Momento IV – Robótica com Java e Programação de Trajetórias}

\begin{itemize}
    \item 02/01/25: Manipulação de motores, sensores e servos com a FTC SDK
    \item 09/01/25: Programação de controle manual (TeleOp) no contexto do robô FTC
    \item 16/01/25: Introdução ao Road Runner e noções básicas de movimentação autônoma
    \item 23/01/25: Programação de trajetórias com Road Runner: conceitos e funções principais
    \item 30/01/25: Ajustes de PID, precisão e tempo nos movimentos autônomos
    \item 06/02/25: Integração entre modos autônomo e manual; estratégias para a competição
    \item 13/02/25: Simulação de partidas e revisão geral dos códigos com foco na FTC
    \item 20/02/25: Preparação final e revisão para a competição
\end{itemize}

Essas ações foram planejadas e executadas de forma a proporcionar aos participantes do projeto uma progressão gradual no aprendizado, abordando desde os conceitos básicos da linguagem de programação Java até tópicos mais avançados relacionados à Programação Orientada a Objetos. Cada encontro (Fig. \ref{fig:4}, Fig. \ref{fig:5} e Fig. \ref{fig:6})  foi estruturado para fornecer uma base sólida de conhecimento e habilidades aos alunos, preparando-os para enfrentar desafios mais complexos no campo da computação e da robótica.

\begin{figure}[H]
    \centering
    \includegraphics[scale=0.22]{imagens/imagem11.png}
    \caption{Início das ações no Centro de Atividades Dionízio Marques de Almeida - Patos - PB.}
    \label{fig:4}
\end{figure}

\begin{figure}[H]
    \centering
    \includegraphics[scale=0.22]{imagens/imagem3.jpeg}
    \caption{Aula sobre Comandos de Decisão utilizando a linguagem Java.}
    \label{fig:5}
\end{figure}

\begin{figure}[H]
    \centering
    \includegraphics[scale=0.22]{imagens/imagem2.jpeg}
    \caption{Aula sobre Herança a linguagem Java.}
    \label{fig:6}
\end{figure}
