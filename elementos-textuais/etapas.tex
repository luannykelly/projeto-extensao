\chapter{ETAPAS DO PROJETO}

O projeto foi organizado em etapas estratégicas e sequenciais, cada uma com objetivos bem definidos e ações planejadas para alcançá-los de forma eficiente. A seguir, descrevem-se essas etapas em detalhes:

\begin{itemize}
    \item \textbf{Levantamento de Necessidades:} A fase inicial consistiu na identificação das principais demandas e dificuldades enfrentadas pelos estudantes no processo de aprendizagem de programação e robótica. Foram conduzidas entrevistas, questionários diagnósticos e observações em sala para mapear os conhecimentos prévios, os interesses dos participantes e os recursos disponíveis na escola;
    
    \item \textbf{Planejamento e Estruturação:} Com base nas informações levantadas, foi realizada a construção do plano de ação do projeto. Nessa etapa, definiram-se os objetivos gerais e específicos, a carga horária, a sequência didática dos conteúdos, as metodologias a serem aplicadas, os critérios de avaliação, o cronograma e os recursos materiais, humanos e tecnológicos necessários para a execução das atividades;
    
    \item \textbf{Desenvolvimento de Conteúdo:} Foram elaborados materiais didáticos interativos voltados para o ensino de programação e robótica com foco no desenvolvimento de competências STEM. Os conteúdos foram adaptados ao público-alvo, incluindo slides, vídeos, apostilas digitais, roteiros de atividades práticas, simuladores online, jogos educativos e protótipos experimentais. Também foram preparados desafios inspirados em competições de robótica;
    
    \item \textbf{Implementação das Ações:} Iniciou-se a execução das aulas teóricas e práticas, seguindo a metodologia ativa. Os estudantes participaram de oficinas, desafios de programação, construção de robôs e resolução de problemas, estimulando o raciocínio lógico, a criatividade e o trabalho em equipe. Também foram realizadas mentorias técnicas e rodas de conversa sobre carreira científica e tecnológica;
    
    \item \textbf{Acompanhamento e Avaliação:} O progresso dos participantes foi monitorado de forma contínua, por meio da observação direta, autoavaliações, resolução de exercícios, devolutivas qualitativas e desafios avaliativos. As avaliações formativas e somativas permitiram identificar avanços cognitivos, engajamento, desenvolvimento de habilidades práticas e dificuldades pontuais, possibilitando ajustes nas estratégias ao longo do percurso;
    
    \item \textbf{Análise de Resultados:} Com a conclusão das atividades, foi realizada uma avaliação global dos resultados obtidos. A análise envolveu a comparação entre os objetivos iniciais e os resultados alcançados, destacando indicadores como melhoria no desempenho, aumento do interesse pelas áreas de ciência e tecnologia e engajamento nas dinâmicas propostas. A equipe também refletiu sobre os desafios enfrentados e as boas práticas identificadas;
    
    \item \textbf{Documentação e Relatório Final:} Todas as fases do projeto foram devidamente registradas, resultando na elaboração de um relatório técnico-científico final. Esse documento sistematiza as ações realizadas, os resultados obtidos, os instrumentos utilizados, os aprendizados coletivos e as recomendações para a replicação e aprimoramento de projetos futuros com escopo semelhante.
\end{itemize}
