\chapter{METODOLOGIA, ESTRATÉGIAS DE AÇÃO, MATERIAL E MÉTODOS}

Neste capítulo, são apresentados a metodologia adotada, as estratégias de ação implementadas, bem como os materiais e métodos utilizados no desenvolvimento do projeto "Prognitive". A integração entre teoria e prática é destacada como elemento essencial do processo, assegurando que os conhecimentos adquiridos fossem contextualizados e aplicados de forma significativa.

Para garantir a eficácia das atividades realizadas, foram implementadas diversas estratégias metodológicas, incluindo:

\begin{itemize}
    \item Aulas expositivas, com ênfase nos conceitos fundamentais de programação, robótica e áreas correlatas, a fim de fornecer uma base teórica sólida aos participantes;
    \item Atividades práticas \textit{hands-on}\footnote{\textit{"Hands-on"} refere-se à experiência prática e direta, em que os participantes lidam ativamente com o conteúdo por meio da experimentação e execução de tarefas. Isso favorece a internalização do conhecimento \cite{graafstra2007hands}.}, utilizando kits de robótica, simuladores e plataformas digitais para a aplicação dos conhecimentos;
    \item Estudos de caso, por meio dos quais os alunos analisaram problemas reais e propuseram soluções baseadas nos conteúdos aprendidos;
    \item Debates em grupo, estimulando a reflexão crítica, o raciocínio argumentativo e o trabalho em equipe;
    \item Resolução de problemas práticos nas áreas de lógica, programação e automação, incentivando a criatividade e a eficiência;
    \item Uso de recursos digitais interativos, como ambientes virtuais de aprendizagem, vídeos, quizzes, tutoriais e jogos educacionais;
    \item Acompanhamento individualizado dos alunos, com orientação contínua, escuta ativa, esclarecimento de dúvidas e devolutivas personalizadas;
    \item Oficinas temáticas e dinâmicas de grupo com foco em criatividade, liderança, pensamento computacional e ética na tecnologia.
\end{itemize}

Essas estratégias foram fundamentadas em teorias pedagógicas sólidas, como a teoria da aprendizagem significativa de Ausubel \cite{ausubel1982aprendizagem, junior2023olhar}, a abordagem socioconstrutivista de Vygotsky \cite{vygotsky1988aprendizagem, paixao2019aprendizagem, cantuaria2023importancia}, e em metodologias ativas contemporâneas de ensino-aprendizagem \cite{kripka2020ensino}, como o \textit{Design Thinking} Educacional, que favorece a empatia, ideação, prototipação e solução de problemas reais. Todo o processo foi adaptado ao perfil dos participantes, considerando seus níveis de conhecimento, motivações, interesses e dificuldades.

Durante a execução do projeto, diversos materiais, ferramentas e recursos foram utilizados para apoiar e potencializar o processo de aprendizagem. Destacam-se:

\begin{itemize}
    \item \textbf{Kits de robótica:} Conjuntos de componentes eletrônicos e mecânicos para montagem de protótipos robóticos;
    \item \textbf{Computadores e dispositivos móveis:} Utilizados para programação, simulação, pesquisa e uso de softwares educacionais;
    \item \textbf{Material impresso:} Apostilas, manuais, guias práticos e roteiros de atividades, que complementaram o conteúdo digital;
    \item \textbf{Recursos de comunicação:} E-mails, fóruns online, plataformas de reuniões e chats de apoio;
    \item \textbf{Simuladores virtuais:} Plataformas como simuladores de linguagem de programação;
    \item \textbf{Softwares educacionais:} Ambientes como Visual Studio Code e plataformas gamificadas;
    \item \textbf{Ambientes físicos preparados:} Salas de aula com acesso à internet, projetores, quadros interativos, mobiliário móvel e climatização;
    \item \textbf{Agenda e cronograma:} Planejamento semanal com datas, temas, atividades e avaliações;
\end{itemize}

Em relação aos métodos pedagógicos, buscou-se adotar abordagens centradas no aluno, com foco na aprendizagem ativa e significativa. Foram utilizados:

\begin{itemize}
    \item \textbf{Aprendizagem baseada em problemas (PBL):} Método centrado em situações-problema para promover autonomia, pesquisa e tomada de decisão;
    \item \textbf{Ensino colaborativo:} Organização de equipes com papéis definidos, promovendo cooperação, comunicação e corresponsabilidade;
    \item \textbf{Aprendizagem ativa:} Metodologias que envolvem diretamente os participantes, como debates, projetos e dinâmicas reflexivas;
    \item \textbf{Feedback construtivo:} Avaliações formativas com devolutivas detalhadas e orientações para melhoria contínua;
    \item \textbf{Gamificação:} Uso de elementos de jogos (desafios, pontuações, recompensas) para aumentar o engajamento;
    \item \textbf{Rodas de conversa e autoavaliação:} Espaços de fala e escuta, nos quais os alunos refletiam sobre o próprio processo de aprendizagem;
    \item \textbf{Aprendizagem baseada em projetos (PjBL):} Desenvolvimento de soluções tecnológicas com etapas de planejamento, execução e apresentação final;
    \item \textbf{Mentorias e tutoria entre pares:} Acompanhamento individual e colaboração entre alunos mais experientes e iniciantes.
\end{itemize}

A diversidade e complementaridade dos métodos adotados contribuiu significativamente para uma experiência de aprendizado rica, contextualizada e alinhada às exigências das competições de robótica e ao desenvolvimento de competências STEM. O uso integrado de diferentes abordagens pedagógicas, associado aos recursos didáticos e tecnológicos, assegurou o cumprimento dos objetivos educacionais propostos e o desenvolvimento integral dos participantes.
