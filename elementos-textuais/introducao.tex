\chapter{INTRODUÇÃO}

O projeto em questão tem como principal objetivo fomentar o aprendizado de programação e robótica entre estudantes do Ensino Médio, por meio de abordagens educacionais inovadoras e metodologias ativas que tornem o processo de ensino mais envolvente e significativo, além de divulgar o curso de ciência da Computação do Câmpus VII da UEPB. 

A partir de ações diversificadas e contextualizadas, busca-se proporcionar uma vivência formativa que una teoria e prática de forma equilibrada e eficiente.

As atividades tiveram início com aulas de lógica de programação, seguida de aulas utilizando a linguagem Java, com ênfase em conceitos de Orientação a Objetos, diretamente aplicados à robótica educacional. Essa integração favorece o desenvolvimento de competências técnicas sólidas, preparando os alunos para enfrentar tanto desafios computacionais quanto práticos, típicos da robótica moderna.

Para reforçar o conteúdo abordado, foram propostos exercícios de fixação que estimulam a aprendizagem ativa, promovendo o raciocínio lógico e o pensamento algorítmico. Essa estratégia contribui para uma assimilação mais eficaz dos conceitos, ao mesmo tempo em que fortalece a autonomia e a capacidade de resolução de problemas dos estudantes.

Durante a execução do projeto, ofereceu-se acompanhamento individualizado, com suporte contínuo na resolução de dúvidas e na realização das atividades práticas. Essa atenção personalizada permitiu respeitar o ritmo de aprendizagem de cada participante, tornando o processo mais inclusivo e eficiente.

Um dos recursos tecnológicos explorados foi o Road Runner, ferramenta que converte o ambiente da arena em um plano cartesiano, otimizando a movimentação do robô durante a fase autônoma da competição. 

O projeto também visa preparar os alunos para a participação na FTC (FIRST Tech Challenge), uma renomada competição nacional e internacional de robótica. Essa etapa final agrega valor à experiência educacional, ao incentivar a aplicação prática do conhecimento adquirido, fortalecer o espírito de equipe e inspirar os estudantes a seguirem carreiras nas áreas de Ciência, Tecnologia, Engenharia e Matemática (STEM).

Com uma proposta estruturada e abrangente, o projeto se destaca por oferecer uma formação prática, desafiadora e motivadora, contribuindo diretamente para preparação de jovens frente às exigências do mundo tecnológico contemporâneo.

 