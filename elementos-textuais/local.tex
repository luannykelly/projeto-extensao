\chapter{LOCAL DE REALIZAÇÃO}

O projeto "Prognitive" foi implementado integralmente na Escola Dionísio Marques de Almeida (DMA), situada na cidade de Patos - PB, com o objetivo de atender aos alunos do Ensino Médio desta instituição, visando uma formação sólida nas áreas de programação e robótica.

A escolha da escola como local de realização do projeto foi estratégica , sendo um dos principais fatores a infraestrutura disponível para o desenvolvimento de atividades práticas. A escola já possuía kits de robótica, o que facilitou a implementação das atividades propostas no projeto. Além disso, a presença de um laboratório de informática bem equipado foi um diferencial crucial, pois permitiu que os alunos utilizassem recursos tecnológicos avançados para aprofundar seus conhecimentos em programação e robótica.

O laboratório de informática foi fundamental para criar um ambiente interativo e dinâmico, essencial para o aprendizado prático de programação. Com computadores modernos e acesso contínuo à internet, os alunos puderam desenvolver suas habilidades através de pesquisas online, prática em linguagens de programação e acesso a materiais educacionais complementares. Esse espaço se tornou um centro de aprendizagem, onde os alunos puderam, além de programar, colaborar em projetos, simular cenários e resolver problemas complexos de maneira colaborativa.

Ademais, a combinação do laboratório de informática com os kits de robótica proporcionou aos alunos uma experiência de aprendizagem prática e envolvente. As aulas foram estruturadas de forma a integrar a construção de robôs físicos com a programação de suas trajetórias e comportamentos, permitindo que os estudantes não só aprendessem a lógica de programação, mas também a aplicassem em um ambiente real, com resultados tangíveis. Essa abordagem prática estimulou a criatividade e a resolução de problemas de maneira inovadora, alinhada às metodologias ativas de ensino.

Portanto, a escolha do local de realização do projeto "Prognitive" foi um fator determinante para o sucesso da iniciativa. A infraestrutura tecnológica disponível, somada ao comprometimento da escola em apoiar o desenvolvimento educacional dos alunos, possibilitou a implementação de um ensino inovador, eficaz e alinhado com as demandas atuais de formação em ciência e tecnologia. O projeto contribuiu significativamente para a promoção do aprendizado em programação e robótica, preparando os alunos para os desafios do futuro.
