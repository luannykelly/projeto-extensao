\chapter{DURAÇÃO DO PROJETO}

O projeto Prognitive teve uma duração total de um ano. Durante esse período, diversas ações e atividades de extensão foram realizadas com o objetivo de promover o aprendizado de programação e robótica, além de engajar os participantes em competições nacionais de tecnologia, como a FIRST Tech Challenge (FTC).

A execução do projeto foi organizada em módulos educacionais que contemplaram desde a introdução à lógica de programação e à linguagem Java até tópicos avançados de programação orientada a objetos. Essas aulas, que ocorreram semanalmente, foram complementadas por workshops e atividades práticas, visando garantir que os participantes não apenas compreendessem os conceitos teóricos, mas também tivessem a oportunidade de aplicar o conhecimento adquirido.

Além das aulas e atividades de ensino, o projeto se dedicou à capacitação prática por meio de oficinas de robótica, focando principalmente no desenvolvimento e programação de robôs. A participação em competições de robótica foi uma das partes mais importantes do projeto, pois proporcionou aos alunos a oportunidade de aplicar as habilidades adquiridas em um ambiente competitivo. A preparação para a competição nacional e internacional foi intensiva, abrangendo desde a montagem e configuração de robôs até o desenvolvimento de algoritmos avançados de controle de movimentos, como o uso do Road Runner para trajetórias autônomas.

Ao longo do projeto, foram realizados momentos de análise e discussão dos resultados das competições, com a finalidade de identificar pontos de melhoria e possibilitar um feedback construtivo.

Em termos de impacto, o projeto "Prognitive" contribuiu significativamente para o desenvolvimento educacional, social e profissional dos participantes. A inclusão de alunos de diferentes níveis de conhecimento em computação e robótica propiciou a democratização do acesso à educação tecnológica de qualidade, além de fomentar o interesse pelas áreas de ciência e inovação.

 O projeto "Prognitive" se mostrou uma experiência enriquecedora tanto para os participantes quanto para a comunidade, com resultados positivos no desenvolvimento de competências técnicas e interpessoais.
