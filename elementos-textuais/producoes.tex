\chapter{PRODUÇÕES CIENTÍFICAS RESULTANTES DO PROJETO DE EXTENSÃO}

Durante o período inicial de execução do "Prognitive", o foco principal esteve na implementação das atividades e na interação com a comunidade. À medida que o projeto avançou, conseguimos alcançar novos patamares, com a produção das primeiras publicações científicas que analisam e documentam os resultados obtidos e os impactos gerados.

O primeiro artigo científico gerado pelo projeto, intitulado "Contributions of the PROGNITIVE Project to the Development of Programming and Robotics Skills in High School", marca um ponto de inflexão significativo na jornada do "Prognitive". Este trabalho analisa o impacto do projeto no desenvolvimento de habilidades de programação e robótica em estudantes do ensino médio, destacando a maneira como a iniciativa tem impulsionado a aprendizagem prática nessas áreas.

\begin{figure}[H]
    \centering
    \caption{Apresentação do artigo "Contributions of the PROGNITIVE Project to the Development of Programming and Robotics Skills in High School" durante o VI Sercomp. }
    \includegraphics[scale=0.20]{imagens/imagem18.jpeg}
\end{figure}

\begin{figure}[H]
    \centering
    \caption{Apresentação do artigo científico sobre o Projeto Prognitive, destacando as contribuições para o ensino de programação e robótica no ensino médio.}
    \includegraphics[scale=0.18]{imagens/imagem21.jpeg.jpg}
    \includegraphics[scale=0.18]{imagens/imagem19.jpeg.jpg}
\end{figure}


Com o sucesso da primeira publicação, planejamos expandir nossas ações para criar novas produções científicas, ampliando a abrangência do projeto com as seguintes iniciativas:
\begin{itemize}
    \item Publicação de artigos científicos que explorem aspectos específicos do "Prognitive", com foco em áreas como ensino de programação, uso de robótica como ferramenta pedagógica.
    \item Produção de livros e manuais para documentar as metodologias e práticas que se mostraram bem-sucedidas ao longo do projeto.
    \item Desenvolvimento de softwares e jogos educativos, como ferramentas para estimular a aprendizagem interativa e engajante nas áreas de programação e robótica.
\end{itemize}


